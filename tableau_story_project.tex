\documentclass[a4paper,11pt]{report}
\usepackage[T1]{fontenc}
\usepackage[utf8]{inputenc}
\usepackage{lmodern}
\usepackage{color}
\usepackage{graphicx}
\graphicspath{ {images/} }
\usepackage{hyperref}

\title{Create a Tableau Story Final Project}
\author{Neil Seas}

\begin{document}

\maketitle
\tableofcontents

\begin{abstract}
This document contains the \textbf{Summary, Design, Feedback, and Resources}
write-ups required by the Udacity Data Analyst Nanodegree as described in the
    \href{https://classroom.udacity.com/nanodegrees/nd002/parts/38e740f3-bea2-43cb-a474-1de2abe3ec4b/modules/982dba08-fe94-413f-91ff-0dce5a6e1e16/lessons/9954c4f7-c621-4e22-baf4-60da1150093c/concepts/656b2cfc-9c83-4d19-931a-af681d43f92c}{Project
    Details}
\end{abstract}

\chapter{Write-Ups}
\section{Summary}
This project investigates cancellations in the year 2008 for airlines in the
United States. The data set was combined with geospatial data from Data.gov to
show patterns of cancellations on a map of the United States.  Further
exploration highlights specific airlines that are consistently poor performers
in terms of cancellation rate.

\href{https://public.tableau.com/profile/neil.seas#!/vizhome/UdacityFinalProject/Story1}{\bf{Link
to version 1}}

\href{https://public.tableau.com/profile/neil.seas#!/vizhome/UdacityFinalProjectv2/Story1}{\bf{Link
to version 2}}

\href{https://public.tableau.com/profile/neil.seas#!/vizhome/UdacityFinalProjectv3/Story1}{\bf{Link
to version 3}}

\section{Design}
My first design decision was to incorporate geospatial data in order to better
visualize how cancellation rates vary across the United States.  In the
visualizations I used both Number of Flights and Cancellation Rate to reflect
the scale of activity.  I believe this choice facilitates quickly identifying
both the most active airports and airports/airlines with relatively higher
cancellation rates.

In the first visualization I chose to use a map to show the cancellation rate by
the airport of origin.  To make this visualization more clean, I chose to filter
the data to only airports with at least 50,000 flights.  I chose a diverging
color palette because the majority of the cancellation rate range is quite
narrow.  The information in the tool tips is provided for airport specific
detail, while the legends are provided to provide information about the meaning
of bubble sizes and color.

For the second visualization I chose a box-plot because I wanted to show both
the distribution of cancellation rates for the carriers, but also to highlight
any outliers.  I chose to also have the size of the dots indicate the size of
the carrier (number of flights).

The third visualization provides similar information to the box-plot, but I felt
that looking at the information in a scatter-plot really highlighted the higher
cancellation rate airlines.  I think it also makes clear that there is a general
trend of larger airlines (by number of flights) having worse performance in
terms of cancellation rate with the clear exception of Southwest.

For the last visualization I returned to using a map to demonstrate that the
poor performers had high cancellation rates regardless of airport.  When
filtered to the six airlines identified in the prior visualizations the impact
of seeing mostly red on the map would allow the reader to quickly draw the
intended conclusion.



One change incorporated from the feedback I received was to remove information
that was not contributing to the desired take-away from the visualization.  In
the case of the maps, this involved limiting the amount of information being
shown (limiting the number of carriers).  I also translated carrier codes into
carrier names for readability.

\section{Feedback}
The feedback I received centered around making the visualizations more concise.
On the maps I was asked to zoom in cluttered regions such as the
northeast.  Also, limiting the visualization to only carriers of interest.  For
the box-plot the reviewer pointed out that I was showing redundant information by
including color to reflect cancellation rate, which was already clearly
reflected by the y-axis.  The reviewer also pointed out that carrier codes
needed to be changed to actual carrier names.

\section{Resources}
    \begin{itemize}
        \item
            \href{https://catalog.data.gov/dataset/usgs-small-scale-dataset-global-map-airports-of-the-united-states-201403-shapefile}{Data.gov}.
        \item \href
            {http://stat-computing.org/dataexpo/2009/the-data.html}{stat-computing.org}
	\item Tableau help
    \end{itemize}
\end{document}
